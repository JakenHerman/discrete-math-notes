\documentclass[10pt,a4paper]{article}
\usepackage[utf8]{inputenc}
\usepackage{amsmath}
\usepackage{amsfonts}
\usepackage{amssymb}
\author{Jaken Herman}
\title{Mathematical Notation, Sets, and Counting}
\begin{document}
\maketitle

\section{Mathematical Notation}
Instead of writing out lengthy text-descriptions for our formulas, it's just easier to use symbols. Here is a list of frequently used symbols in discrete mathematics:
\begin{itemize}
\item $\mathbb{R}$: Set of all real numbers.
\item $\mathbb{Q}$: Set of all rational numbers.
\item $\mathbb{Z}$: Set of all integers.
\item $\mathbb{Z_+}$: Set of all non-negative integers.
\item $\mathbb{N}$: Set of all natural numbers.
\item $\in$: Element of.
\end{itemize}

Knowing these symbols, we can now mathematically describe $\mathbb{Z_+}$ as:
\linebreak \center $\mathbb{Z_+} = \{ x \in \mathbb{Z} : x \geq 0 \}$
\linebreak
\linebreak
The "$:$" in that statement basically means 'satisfying the following condition'. In other words, if we read the formula aloud, you would say something like "$\mathbb{Z_+}$ equals every value of x in $\mathbb{Z}$ where x is greater than or equal to 0."
\end{document}
